\documentclass{pset}
\usepackage{ctex}

\name{廖星宇}
\email{sherlockliao01@gmail.com}

\course{周志华-机器学习}
\instructor{}
\assignment{Problem Set chapter 3}
\duedate{}

\begin{document}

\begin{problem}%1{{{
\textbf{3.7} 令码长为9,类别为4,试给出海明距离意义下理论最优的ECOC二元码并证明之。\\

\textbf{Solution}:


\end{problem}

\begin{problem}
\textbf{3.8} ECOC编码能起到理想纠错作用的重要条件是:在每一位编码上出错的概率相当且独立。试析多分类任务经ECOC编码后产生的二类分类器满足该条件的可能性及由此产生的影响。


\end{problem}

\begin{problem}
\textbf{3.9} 使用OvR和MvM将多分类任务分解为二分类任务求解是,试述为何无需专门针对类别不平衡性进行处理。

\textbf{Solution:}

对于OvR和MvM而言,每次的数据集都是不均衡的,但是其会遍历每一个类,这样类别的不均衡性在遍历的过程中就抵消了。比如一共有4类,$c_1, c_2, c_3, c_4$,第一次取$c_1$为正样本,其余为负样本,这时正样本$c_1$的数目就少,但是第二次将$c_2$作为正样本时,$c_1$就变成了负样本,这样对于分类器$c_1$的样本又变多了,所以不断地遍历,使得每一种样本都能够被取到,这样就不用进行不平衡处理。
\end{problem}

\begin{problem}
\textbf{3.10} 试推导多分类代价敏感学习(仅考虑基于类别的误分类代价)使用“在放缩”能获得理论最优解的条件。


\end{problem}
\end{document}
